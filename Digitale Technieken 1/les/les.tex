\documentclass[11pt, a4paper]{article}
\usepackage{tikz, tikz-timing, verbatim}
\usepackage[american]{circuitikz}
\usetikzlibrary{positioning}
\usepackage[dutch, english]{babel}
\usepackage[linkcolor=black,urlcolor=blue,citecolor=black]{hyperref}
\usepackage[font=small,format=plain,labelfont=bf,up,textfont=it,up]{caption}
\usepackage[usenames,dvipsnames]{pstricks}
\usepackage[parfill]{parskip}
\hypersetup{colorlinks=true}
\usepackage{epsfig, amsmath, epic, eepic, float, subfig, amsfonts, color, amsthm, textcomp, microtype, graphicx}
\newcommand{\HRule}{\rule{\linewidth}{0.5mm}}

\begin{document}

\selectlanguage{dutch}
\begin{titlepage}
\begin{center}
\includegraphics[width=0.5\textwidth]{./logo.pdf}~\\[1cm]


\textsc{\Large Digitale Technieken 1}\\[0.5cm]

\HRule \\[0.4cm]
{ \LARGE \bfseries Lesnota's}\\[0.4cm]
{\large \textit{gegeven door D.Claus}}\\[0.2cm]

\HRule \\[1.5cm]

\begin{minipage}{0.4\textwidth}
\begin{flushleft} \large
\emph{Door:}\\
Haroen \textsc{Viaene}\\

\end{flushleft}
\end{minipage}
\begin{minipage}{0.4\textwidth}
\begin{flushright} \large
\large{1$^{\text{ste}}$ fase bachelor Elektronica-ICT}\\
\end{flushright}
\end{minipage}

\vfill

{\large 2014-2015}

\end{center}
\end{titlepage}

\newpage

\section*{Inhoud}

\tableofcontents

\newpage

\section{Inleiding}

\section{Les 1}

\subsection{evaluatie}

analoog aan ElekSign, maar 6 i.p.v. 8 SP

volgend jaar wordt oefeningen en theorie samengevoegd.

\subsection{inleiding}

\subsubsection{analoge weergave}

Analoog wil zeggen dat het continu, vloeiend is.

Voorbeeld: analoge thermometer, draaispoelmeter (stroomsterkte), weeghaak\dots

\subsubsection{digitale weergave}

sprongsgewijs, discontinu.

secondewijzer, verbreken van een contact\dots

\subsubsection{digitale schakelingen}

Digitale schakelingen maken gebruik van logische poorten.

een spanning wordt omgezet naar 0 of 1 (0-5 bv), negatieve logica kan ook (0 is hoog, 1 is laag), maar wordt niet gebruikt.

\subsection{\dots}

\subsection{Karnaugh-kaart}

\begin{tabular}{c c | c | c | c | c |}
  \_ & \_ & \_ & BC & \_ & \_\\
  \_ & \_ & 00 & 01 & 11 & 10 \\
   \hline
   \_ & 0 &  0 & 1 & 1 & 0 \\
   \hline
   \_ & 1 &  1 & 0 & 0 & 1 \\
   \hline
\end{tabular}

\begin{verbatim}
    	   BC
     00 01 11 10
    +--+--+--+--+
  0 |0 |1 |1 |0 |
 A  +--+--+--+--+
  1 |1 |0 |0 |1 |
    +--+--+--+--+
\end{verbatim}


\newpage

\section{Les 2}

\subsection{Symbolen}

haakje voor NOT

$$\& = AND$$

$$\le1 = OR$$

$$=1 = XOR$$

$$2k = even$$

\subsection{Oefeningen}

\subsubsection{XOR}

IEC = =1

waarheidstabel:

\begin{tabular}{c c c || r}
A & B & C & Y \\
\hline
0 & 0 & 0 & 0 \\
0 & 0 & 1 & 1 \\
0 & 1 & 0 & 1 \\
0 & 1 & 1 & 0 \\
1 & 0 & 0 & 1 \\
1 & 0 & 1 & 0 \\
1 & 1 & 0 & 0 \\
1 & 1 & 1 & 0 \\
\end{tabular}

(in de praktijk: 1 (door te gebruiken van 2 XOR-poorten met 2 ingangen))

Karnaugh-kaart

\begin{tabular}{c c | c | c | c | c |}
  \_ & \_ & \_ & BC & \_ & \_\\
  \_ & \_ & 00 & 01 & 11 & 10 \\
   \hline
   \_ & 0 &  0 & 1 & 1 & 1 \\
   \hline
   \_ & 1 &  1 & 0 & 0 & 1 \\
   \hline
\end{tabular}

\begin{verbatim}
        BC
    00 01 11 10
   +--+--+--+--+
 0 |0 |1 |0 |1 |
A  +--+--+--+--+
 1 |1 |0 |0 |1 |
   +--+--+--+--+
\end{verbatim}

Pulsdiagram:

\begin{tikztimingtable}
  A   &  LHLHLHLH \\ % ends with edge
  B   &  LLHHLLHH \\ % starts with edge
  C   &  LLLLHHHH \\ % ends with edge
  Y   &  LHHLHLLL \\
\end{tikztimingtable}

$$Y=AB+\overline{(AB) \oplus C}$$

\begin{verbatim}
A ----
      &--------
B ----   |    |
         |    -------
         ---        ≤1\---Y
           =1--------
C ----------
\end{verbatim}

\begin{circuitikz} \draw
(0,2) node[and port] (and) {}
(2,1) node[or port] (or) {}
(and.in 1) node[anchor=east] {A}
(and.in 2) node[anchor=east] (bnode) {B}
(and.out) -| (or.in 1)
node[below=of bnode] (cnode) {C}
(cnode) -| (or.in 2)
;\end{circuitikz}

\newpage

\section{Les 3}

\subsection{Sum of Products}

\begin{tabular}{l l l || c}
A & B & X  & MINTERM \\
\hline
0 & 0 & 0  & $\overline{A}\cdot \overline{B}$ \\
0 & 1 & 1  & $\overline{A}\cdot B$ \\
1 & 0 & 1  & $A\cdot \overline{B}$ \\
1 & 1 & 0  & $A\cdot B$ \\
\end{tabular}


$$X = A\cdot \overline{B} + \overline{A}\cdot B = A  \oplus  B$$


\subsection{Product of Sums}

\begin{tabular}{l l l || c}
A & B & X & MAXTERM \\
\hline
0 & 0 & 0 & $A+B$ \\
0 & 1 & 1 & $A+\overline{B}$ \\
1 & 0 & 1 & $\overline{A}+B $\\
1 & 1 & 0 & $\overline{A}+\overline{B}$ \\
\end{tabular}

$$X = A+\overline{B} \cdot \overline{A}+B $$

%%$$X = {M \prod{\overline{(0,3)}} $$


\subsection{Implementatie van logische functies}

$$ X = \overline{A}BC + $$

\newpage

\end{document}