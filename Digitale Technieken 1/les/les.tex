\documentclass[11pt, a4paper]{article}
\usepackage[dutch, english]{babel}
\usepackage{graphicx}
\usepackage[linkcolor=black,urlcolor=blue,citecolor=black]{hyperref}
\usepackage[font=small,format=plain,labelfont=bf,up,textfont=it,up]{caption}
\usepackage[usenames,dvipsnames]{pstricks}
\usepackage[parfill]{parskip}
\hypersetup{colorlinks=true}
\usepackage{epsfig, amsmath, epic, eepic, float, subfig, amsfonts, color, amsthm, textcomp, microtype, fullpage}
\usepackage[all]{xy}
\newcommand{\HRule}{\rule{\linewidth}{0.5mm}}

\begin{document}

\selectlanguage{dutch}
\begin{titlepage}
\begin{center}
\includegraphics[width=0.5\textwidth]{./logo.pdf}~\\[1cm]


\textsc{\Large Digitale Technieken 1}\\[0.5cm]

\HRule \\[0.4cm]
{ \LARGE \bfseries Lesnota's}\\[0.4cm]
{\large \textit{gegeven door D.Claus}}\\[0.2cm]

\HRule \\[1.5cm]

\begin{minipage}{0.4\textwidth}
\begin{flushleft} \large
\emph{Door:}\\
Haroen \textsc{Viaene}\\

\end{flushleft}
\end{minipage}
\begin{minipage}{0.4\textwidth}
\begin{flushright} \large
\large{1$^{\text{ste}}$ fase bachelor Elektronica-ICT}\\
\end{flushright}
\end{minipage}

\vfill

{\large 2014-2015}

\end{center}
\end{titlepage}

\newpage

\section*{Inhoud}

\tableofcontents

\newpage

\section{Inleiding}

\section{Les 1}

\subsection{evaluatie}

analoog aan ElekSign, maar 6 i.p.v. 8 SP

volgend jaar wordt oefeningen en theorie samengevoegd.

\subsection{inleiding}

\subsubsection{analoge weergave}

Analoog wil zeggen dat het continu, vloeiend is.

Voorbeeld: analoge thermometer, draaispoelmeter (stroomsterkte), weeghaak...

\subsubsection{digitale weergave}

sprongsgewijs, discontinu.

secondewijzer, verbreken van een contact...

\subsubsection{digitale schakelingen}

Digitale schakelingen maken gebruik van logische poorten.

een spanning wordt omgezet naar 0 of 1 (0-5 bv), negatieve logica kan ook (0 is hoog, 1 is laag), maar wordt niet gebruikt.

\subsection{...}

\subsection{Karnaud-kaart}


\begin{verbatim}
    	   BC
     00 01 11 10
    +--+--+--+--+
  0 |0 |1 |1 |0 |
 A  +--+--+--+--+
  1 |1 |0 |0 |1 |
    +--+--+--+--+
\end{verbatim}


\newpage

\section{Les 2}

bla bla bla

\newpage

\section{Les 3}

bla bla bla

\newpage

\end{document}