\documentclass[11pt, a4paper]{article}
%\usepackage{tikz-timing}
\usepackage[dutch, english]{babel}
\usepackage{graphicx}
\usepackage[linkcolor=black,urlcolor=blue,citecolor=black]{hyperref}
\usepackage[font=small,format=plain,labelfont=bf,up,textfont=it,up]{caption}
\usepackage[usenames,dvipsnames]{pstricks}
\usepackage[parfill]{parskip}
\hypersetup{colorlinks=true}
\usepackage{epsfig, amsmath, epic, eepic, float, subfig, amsfonts, color, amsthm, textcomp, microtype, fullpage}
\usepackage[all]{xy}
\newcommand{\HRule}{\rule{\linewidth}{0.5mm}}

\begin{document}

\selectlanguage{dutch}
\begin{titlepage}
\begin{center}
\includegraphics[width=0.5\textwidth]{./logo.pdf}~\\[1cm]


\textsc{\Large Digitale Technieken 1}\\[0.5cm]

\HRule \\[0.4cm]
{ \LARGE \bfseries Lesnota's}\\[0.4cm]
{\large \textit{gegeven door D.Claus}}\\[0.2cm]

\HRule \\[1.5cm]

\begin{minipage}{0.4\textwidth}
\begin{flushleft} \large
\emph{Door:}\\
Haroen \textsc{Viaene}\\

\end{flushleft}
\end{minipage}
\begin{minipage}{0.4\textwidth}
\begin{flushright} \large
\large{1$^{\text{ste}}$ fase bachelor Elektronica-ICT}\\
\end{flushright}
\end{minipage}

\vfill

{\large 2014-2015}

\end{center}
\end{titlepage}

\newpage

\section*{Inhoud}

\tableofcontents

\newpage

\section{Inleiding}

\section{Les 1}

\subsection{evaluatie}

analoog aan ElekSign, maar 6 i.p.v. 8 SP

volgend jaar wordt oefeningen en theorie samengevoegd.

\subsection{inleiding}

\subsubsection{analoge weergave}

Analoog wil zeggen dat het continu, vloeiend is.

Voorbeeld: analoge thermometer, draaispoelmeter (stroomsterkte), weeghaak\dots

\subsubsection{digitale weergave}

sprongsgewijs, discontinu.

secondewijzer, verbreken van een contact\dots

\subsubsection{digitale schakelingen}

Digitale schakelingen maken gebruik van logische poorten.

een spanning wordt omgezet naar 0 of 1 (0-5 bv), negatieve logica kan ook (0 is hoog, 1 is laag), maar wordt niet gebruikt.

\subsection{\dots}

\subsection{Karnaud-kaart}


\begin{verbatim}
    	   BC
     00 01 11 10
    +--+--+--+--+
  0 |0 |1 |1 |0 |
 A  +--+--+--+--+
  1 |1 |0 |0 |1 |
    +--+--+--+--+
\end{verbatim}


\newpage

\section{Les 2}

\subsection{Symbolen}

haakje voor NOT

$$\& = AND$$

$$\le1 = OR$$

$$=1 = XOR$$

$$2k = even$$

\subsection{Oefeningen}

\subsubsection{XOR}

IEC = =1

waarheidstabel:

\begin{tabular}{c c c || r}
A & B & C & Y \\
\hline
0 & 0 & 0 & 0 \\
0 & 0 & 1 & 1 \\
0 & 1 & 0 & 1 \\
0 & 1 & 1 & 0 \\
1 & 0 & 0 & 1 \\
1 & 0 & 1 & 0 \\
1 & 1 & 0 & 0 \\
1 & 1 & 1 & 0 \\
\end{tabular}

(in de praktijk: 1 (door te gebruiken van 2 XOR-poorten met 2 ingangen))

Karnaug-kaart

\begin{verbatim}
        BC
    00 01 11 10
   +--+--+--+--+
 0 |0 |1 |0 |1 |
A  +--+--+--+--+
 1 |1 |0 |0 |1 |
   +--+--+--+--+
\end{verbatim}

Pulsdiagram:

\begin{verbatim}
A = _-_-_-_-
B = __--__--
C = ____----
Y = _--_-___
\end{verbatim}

%\texttiming{HLHLHLHLHLHL}

% \begin{tikztimingtable}
%   Coarse Pulse                          & 3L 16H 6L \\
%   Coarse Pulse - Delayed 1              & 4L 16H 5L \\
%   Coarse Pulse - Delayed 2              & 5L 16H 4L \\
%   Coarse Pulse - Delayed 3              & 6L 16H 3L \\
%   \\ % Gives vertical space
%   Final Pulse Set                       & 3L 16H 6L \\
%   Final Pulse $\overline{\mbox{Reset}}$ & 6L 16H 3L \\
%   Final Pulse                           & 3L 19H 3L \\
% \end{tikztimingtable}

$$Y=AB+\overline{(AB) \oplus C}$$

\begin{verbatim}
A ----
      &--------
B ----   |    |
         |    -------
         ---        ≤1\---Y
           =1--------
C ----------
\end{verbatim}

\newpage

\section{Les 3}

\subsection{Sum of Products}

\begin{verbatim}
A B X MINTERM
0 0 0 \overline{A}.\overline{B}
0 1 1 \overline{A}.B
1 0 1 A.\overline{B}
1 1 0 A.B
\end{verbatim}

$$X = A.\overline{B} + \overline{A}.B = A  \oplus  B$$


\subsection{Product of Sums}

\begin{verbatim}
A B X MAXTERM
0 0 0 A+B
0 1 1 A+\overline{B}
1 0 1 \overline{A}+B
1 1 0 \overline{A}+\overline{B}
\end{verbatim}

$$X = A+\overline{B} \cdot \overline{A}+B $$

%%$$X = {M \prod{\overline{(0,3)}} $$


\subsection{Implementatie van logische functies}

$$ X = \overline{A}BC + $$

\newpage

\end{document}