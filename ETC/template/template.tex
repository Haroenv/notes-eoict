\documentclass[11pt, a4paper]{article}
\usepackage[dutch, english]{babel}
\usepackage[pdftex]{graphicx}
\usepackage[linkcolor=black,urlcolor=blue,citecolor=black]{hyperref}
\usepackage[font=small,format=plain,labelfont=bf,up,textfont=it,up]{caption}
\usepackage[usenames,dvipsnames]{pstricks}
\usepackage[parfill]{parskip}
\hypersetup{colorlinks=true}
\usepackage{epsfig,amsmath, epic, eepic, float, subfig, amsfonts, color, amsthm, textcomp, microtype, fullpage}
\usepackage[all]{xy}
\newcommand{\HRule}{\rule{\linewidth}{0.5mm}}
\DeclareMathOperator{\bgtan}{bgtan}

\begin{document}

\selectlanguage{dutch}
\begin{titlepage}
\begin{center}
\includegraphics[width=0.5\textwidth]{./logo.pdf}~\\[1cm]


\textsc{\Large Vak}\\[0.5cm]

\HRule \\[0.4cm]
{ \LARGE \bfseries Titel}\\[0.4cm]
{\large \textit{ondertitel}}\\[0.2cm]

\HRule \\[1.5cm]

\begin{minipage}{0.4\textwidth}
\begin{flushleft} \large
\emph{Door:}\\
Haroen \textsc{Viaene}\\

\end{flushleft}
\end{minipage}
\begin{minipage}{0.4\textwidth}
\begin{flushright} \large
\large{1$^{\text{ste}}$ fase bachelor Elektronica-ICT}\\
\end{flushright}
\end{minipage}

\vfill

{\large 2014-2015}

\end{center}
\end{titlepage}

\newpage

\section*{Inhoud}

\tableofcontents

\newpage

\section{Inleiding}

\section{Inhoud}

bla bla bla

\newpage

\begin{thebibliography}{9}
	
\bibitem{websiteZoveel}
	\textbf{Bron}, 
	Titel 
	[raadpleging]. 
	\url{http://website.adres/pagina}

\bibitem{boekZoveel}
	\textbf{Auteur}, 
	Titel, 
	Uitgever

\end{thebibliography}

\end{document}  
